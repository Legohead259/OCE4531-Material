% ===========================================
% Data Processing
% Written by: Wensen Liu
%
% Date: 07/20/2022
% Last Revision: 07/20/2022
% ============================================

\setchapterstyle{kao}
\chapter{Software Basics and Architecture}
\setchapterpreamble[u]{\margintoc}
\labch{software_basics_and_arch}
% \addcontentsline{toc}{chapter}{Data Processing} % Add the preface to the table of contents as a chapter

While the world of physical hardware can be harnassed to develop systems that accomplish tasks, oftentimes these implementations become prohibitively complex as the scale of
implementation grows larger. For example, a modern computer processor contains billions of transistors and it would be impossible to set and reset each one of these transistors
by hand. Thankfully, as you probably already know, modern programming consists of several layers of abstraction above the transistor level, which allow us to develop meaningful
software solutions while leveraging the expansive amount of hardware that we have access to.

This chapter will focus on introducing you to the fundamentals of hardware-based software development. You may find that several if not all of these topics are also covered in
introductory programming courses. However, since we do not presume that everyone who has taken this class has had background programming experience, we will cover all the topics
necessary for this course from the ground up.

\section{Arduino Basics} \labsec{arduino_basics}
Lorem Ipsum Dolores

\section{Primitive Programming Datatypes} \labsec{primitive_programming_datatypes}
The baseline for every programming language is an understanding of datatypes.
Datatypes are how we can classify the characteristics of the blocks of information that our program will process.
These blocks of information can be collected data from the environment, information that the program stores internally, function inputs/outputs, etc.
Additional information on arduino datatypes (and in general, most arduino syntax) can be found on
\href{https://www.tutorialspoint.com/arduino/arduino_data_types.htm}{tutorialspoint} or the \href{https://www.arduino.cc/reference/en/}{Arduino Language Reference}
\marginnote{For the following sections, example code blocks will be given. As such, please note that the \enquote{$\backslash\backslash$} symbol is used to denote a \enquote{comment} and therefore
is not code that will execute and that the capitalization of keywords is deliberate (since Arduino C is case sensitive)}

    \subsection{void}
    The most basic of datatypes, the \textbf{void} keyword is used to represent \enquote{nothing}. For arduino programming, it is primarily used to denote that a function
    does not give us back any output. Do note, however, that the \textbf{void} datatype cannot be applied to variables.
    \marginnote{We will go more depth into functions in a later section, but for the purposes of this section you can think of functions as boxes that produce output from
    a given input. In the case of \textbf{void} we have no output.}
    
    \paragraph*{Proper usage of Void} The following code block demonstrates the proper usage of the \textbf{void} datatype.
    \begin{lstlisting}[linewidth=1.5\textwidth, language=C++]
        void function_name (\\ input variables) {
            \\ function code
            \\ note how there is no return statement (we are outputting nothing)
        }\end{lstlisting}

    \paragraph*{Improper usage of void} The following code block demonstrates the improper usage of the \textbf{void} datatype.
    \begin{lstlisting}[linewidth=1.5\textwidth, language=C++]
        void variable_name = \\ some value;\end{lstlisting}
    
    \subsection{bool}
    Boolean datatypes are used to store logic values, in this case the logic values of \textbf{True} or \textbf{False}. The \textbf{bool} keyword is used in both function and
    variable declarations to classify the function output or variable type as boolean logic.

    \paragraph*{Boolean usage} The following code block demonstrates how the \textbf{bool} keyword is used in a function declaration and a variable declaration
    \begin{lstlisting}[linewidth=1.5\textwidth, language=C++]
        bool function_name (\\ input variables) {
            bool variable_name = True;
            return variable_name;
        }\end{lstlisting}

    \subsection{Char}
    Character datatypes are special when programming in Arduino C. While they do represent what you would conventionally think as a character (for example, the letter \enquote{a}),
    they are defined based on their ascii value (\href{https://en.cppreference.com/w/cpp/language/ascii}{ASCII Reference}), which allow you to manipulate many more characters than
    what can be typed by a keyboard.

    The \textbf{char} keyword is used in variable and function declarations to classify the result as a character, and there are two ways to syntaxically represent a character.
    \paragraph*{Method 1: Using Single Quotes} The first form of syntax makes more sense visually, and involves wrapping a typed character within two single quotes
    \begin{lstlisting}[linewidth=1.5\textwidth, language=C++]
        char variable_name = 'a';\end{lstlisting}
    Do note, however, that this method restricts you to the characters that can be typed using a conventional keyboard. In addition, when declaring a character in this form, you
    cannot wrap multiple characters inside the quotes.
    \begin{lstlisting}[linewidth=1.5\textwidth, language=C++]
        char variable_name = 'abc'; \\ this is improper declaration \end{lstlisting}
    \paragraph*{Method 2: Using ASCII Values} As mentioned previously, characters are encoded via ASCII values when used in a programming context. As such, we can define a
    character using its corresponding ASCII value.
    \begin{lstlisting}[linewidth=1.5\textwidth, language=C++]
        char variable_name = 97; \\ ASCII value 97 corresponds to character 'a'\end{lstlisting}
    Since one character is equivalent to one byte of data, the maximum decimal value that can be used in a character declaration is 255.

    \begin{kaobox}[frametitle=Aside: Some quirks with characters]
        For all intents and purposes, characters are 8-bit binary numbers (0-255 in decimal). They become \enquote{characters} based on how they are interpreted when output to
        a display. Because of this dual interpretation, you can manipulate characters as if they were numbers. For example
        \begin{equation*}
            'a' + 1 = 'b'
        \end{equation*}
        However, if you were to use \enquote{1} (the character representation of 1) as the input you would get
        \begin{equation*}
            'a' + '1' = 97 + 49 = 156
        \end{equation*} 
        Which, for those that are curious, 156 represents the ASCII character \oe. In addition, there are special characters that can be typed using the single quotes 
        method that do not strictly follow the \enquote{only one character} rule. In particular, the $'\backslash n'$ (newline) and $'\backslash t'$ (tab) characters are often used.
    \end{kaobox}

    \subsection{unsigned char and byte}
    The unsigned character and byte datatypes are both datatypes that are used to represent a byte of data (8 bits, 0-255 in decimal). While they share the same range of values
    as the char datatype, they should not be used interchangeably with the char datatype. The \textbf{unsigned char} and \textbf{byte} keywords are used to define function output
    and variable values.
    \marginnote{Generally speaking, it is conventional to use the \textbf{byte} keyword over the \textbf{unsigned char} keyword for both brevity and code readability}
    \paragraph*{unsigned char and byte usage} the following block of code demonstrates how to use the \textbf{unsigned char} and \textbf{byte} keywords
    \begin{lstlisting}[linewidth=1.5\textwidth, language=C++]
        byte function_name (\\ input variables) {
            byte variable_name_1 = 0;
            unsigned char variable_name_2 = 128;
            return variable_name_1;
        }\end{lstlisting}

    \subsection{int and short}
    Both the \textbf{int} datatype and the \textbf{short} datatype are used to store 16-bit (2 byte) signed integers. This makes their effective range -32,768 to 32,767. By convention, the \textbf{int} datatype
    is the more commonly used one in variable and output declaration.
    \marginnote{One of the things that you will have to look out for when using numeric datatypes is \textbf{overflow}; ie. when the number passed into a data field is larger than the range of that datatype.
    Overflow will \textbf{not} stop the code from compiling, but the code's behavior will arbitrarily differ from what you expect.}
    \paragraph*{int and short usage} the following block of code demonstrates how to use the \textbf{int} and \textbf{short} keywords
    \begin{lstlisting}[linewidth=1.5\textwidth, language=C++]
        int function_name (\\ input variables) {
            int variable_name_1 = -20000;
            short variable_name_2 = 12763;
            return variable_name_1;
        }\end{lstlisting}
    
    \subsection{unsigned int and word}
    Similar to the case with \textbf{char} and \textbf{unsigned char}, there are also unsigned variants of the \textbf{int} and \textbf{short} datatypes respectively known as \textbf{unsigned int} and
    \textbf{word}. These datatypes are both 16-bit(2 byte) integers; however, they span only the \textbf{positive} range of 16-bit integers (0 to 65,535).
    \marginnote{The \textbf{unsigned int} datatype has \textbf{known} overflow behavior unlike \textbf{int}. When the value of an unsigned integer goes below zero, it wraps around to the max of the
    range (so -1 = 65535, -2 = 65534) and when it exceeds 65535, it wraps back to zero (so 65536 = 0, 65537 = 1)}
    \paragraph*{unsigned int and word usage} the following block of code demonstrates how to use the \textbf{unsigned int} and \textbf{word} keywords
    \begin{lstlisting}[linewidth=1.5\textwidth, language=C++]
        unsigned int function_name (\\ input variables) {
            unsigned int variable_name_1 = 1203;
            word variable_name_2 = 55000;
            return variable_name_1;
        }\end{lstlisting}
    
    \subsection{long and unsigned long}
    We had previously covered that a \textbf{short} is a 16-bit integer. Therefore, appropriately, a \textbf{long} is a 32-bit signed integer value encompassing the range -2,147,483,648 to 2,147,483,647. As
    per convention, the \textbf{unsigned long} store a 32-bit unsigned integer value with range 0 to 4,294,967,295.
    \paragraph*{long and unsigned long usage} the following block of code demonstrates how to use the \textbf{long} and \textbf{unsigned long} keywords
    \begin{lstlisting}[linewidth=1.5\textwidth, language=C++]
        long function_name (\\ input variables) {
            long variable_name_1 = -2000000;
            unsigned long variable_name_2 = 1000000;
            return variable_name_1;
        }\end{lstlisting}
    
    \subsection{float and double}
    There are also datatypes that are used to store decimal values vs. pure integer, these are the \textbf{float} and \textbf{double} datatypes. For the boards that you will be using, the \textbf{float} and
    the \textbf{double} datatypes are functionally the same. They both cover the range -3.4028235E+38 to 3.4028235E+38, and are both stored in memory as 32-bit values. 
    \paragraph*{float and double usage} the following block of code demonstrates how to use the \textbf{float} and \textbf{double} keywords
    \begin{lstlisting}[linewidth=1.5\textwidth, language=C++]
        float function_name (\\ input variables) {
            float variable_name_1 = -300.21;
            double variable_name_2 = 600.123;
            return variable_name_1;
        }\end{lstlisting}
    \paragraph*{Floating point error} One of the issues that you may run into when dealing with floats and doubles is the floating point (roundoff) error. Due to how floating point numbers are represented
    (for more info on that, you can read up on the \href{https://en.wikipedia.org/wiki/IEEE_754}{IEEE-754 Standard}), and due to the fact that floating point numbers are stored in a finite number of bits (32),
    floating point numbers are \textbf{not absolutely accurate} and will differ from the true value of a datapoint by a small amount. For the majority of applications, this slight difference between 
    actual values and real values is trivial and non-impactful to the application. However, for applications that store small magnitude numbers as data, or rely on high-precision calculations, floating
    point error is something to watch out for.
    
    \subsection{Arrays}
    Arrays are one of the most fundamental structures to understand when working with any programming language. In Arduino C, they are treated as a block of memory that is broken up into chunks, where each
    chunk stores a single value of a datatype. Arrays in Arduino follow an explicit set of rules that govern how they are defined and used, which we will cover below.
    \paragraph*{Rule 1: Arrays can only hold 1 data type} Unlike some other languages, an array in Arduino (and in C in general) must be defined to store a single datatype. You cannot have an array that has
    both int and float datatypes for example. For the reasoning behind this rule, you can take a look at the following aside.
    \paragraph*{Rule 2: Arrays must have a predefined number of values} When you initialize an array, you must tell the program how many values are going to be stored in the array. After this declaration, you 
    cannot modify the size of that array. The reasoning behind this rule is also covered in the following aside.
    \begin{kaobox}[frametitle=Aside: The weird world of memory allocation]
        C (and its variants) are programming languages that deal directly with memory. While this makes them faster than languages that do not directly modify memory, it also restricts how flexible variables
        and memory structures can be when using them. When reading or writing to memory, the amount of memory you use and how that memory will be divided up must be explicitly defined. If the variable asks 
        for 32 bits of memory, it will get 32 bits of memory, no more, no less. In a similar sense, arrays also must explicitly define how much memory they will use. An array that wants to store 10 ints 
        (16-bit) will be allocated $10 * 16 = 160$ bits of memory divided into 10x 16-bit chunks. If you tried to store a long (32-bit) into one of those chunks, it will not fit and the program will error. Now,
        while it is possible to allocate a raw chunk of memory to be manually partitioned and used, how to do that is beyond the scope of this class. Those interested can read up on \textbf{memory allocation
        in C}
    \end{kaobox}